\documentclass[12pt]{article}
\usepackage{csvsimple}

\title{Percentage of Water in Popcorn}
\author{Zach Latta, Angel Madrid, Bailey Frierson, Samantha Lee, Griffin Troy}

\begin{document}

\maketitle

\section{Data Table}
\csvautotabular{masses.csv}

\section{Questions}

\subsection{Lab Safety}

\begin{enumerate}
  \item What lab safety precautions should be taken when working with Bunsen Burners?
    \begin{itemize}
      \item Wear safety goggles. Avoid touching the upper area of the tube, it
        is hot. Verify that the gas is working properly.
    \end{itemize}
  \item Name the various pieces of laboratory that were used in the lab today.
    \begin{itemize}
      \item Bunsen Burner
      \item Tongs
      \item Gas
      \item Ringstand
      \item Iron ring
      \item Wire gauze
    \end{itemize}
\end{enumerate}

\subsection{Analysis}

\begin{enumerate}
  \item Determine the mass of 10 unpopped popcorn kernels for each brand of
    popcorn.
  \item Determine the mass of 10 popped popcorn kernels for each brand of popcorn.
  \item Determine the mass of water that was lost when the popcorn popped for each brand.
\end{enumerate}

\subsection{Conclusions}

\begin{enumerate}
  \item Determine the percentage by pass of water in each brand of popcorn.
  \item Do all brands of popcorn contain approximately the same percentage of water?
  \item What are some likely areas of error in this experiment?
\end{enumerate}

The percentage of water in poopcorn can be dtermined by the following equation.

Initial mass-final mass X100 = percentage of H2O in upopped popcorn initial mass

\end{document}
